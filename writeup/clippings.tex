Stemming from Grices's maxims (be clear, concise, relevant, and truthful) \cite{grice1975}

Young children show early evidence of expecting speakers to communicate cooperatively to expand common ground \cite{clark1996}.  

control condition in \citeA{gelman2003}


Children's appreciation for linguistic framing choices suggests that they are sensitive to how and when cultural information is transmitted.  They pick up on when speakers provide cues to generalizable, conventional information versus more narrow or individualized references. 

Even without direct labeling, children expect demonstrations to convey complete information: children given a new toy and told, ``This is my toy. I'm going to show you how my toy works. Watch this!'' and a squeaking action were more likely to only use the toy to perform this function than participants who were shown the same action accidentally (``Huh! Did you see that?'') \cite{bonawitz2011}.  Pedagogical demonstrations also help children make more inductive inferences about shared functions of other category members \cite{butler2012, butler2014}.  Together, these findings suggest that children appear to privilege pedagogical information as conveying cultural expectations about conventional behaviors. 

Children's actions convey sophisticated reasoning from ostensive demonstrations.  In one study, 12- to 14-month-olds selected a preference for either pink or black lollipops, then were shown two cups, one that contained mostly pink lollipops and one that contained mostly black lollipops.  An experimenter closed her eyes and selected a lollipop from each cup and placed each in individual opaque containers so that the color was never seen.  Although there was a chance of getting the either flavor from either cup, babies were more likely to select the lollipop from the sample that favored their desired outcome. Young children use sampling likelihoods to infer other peopleÕs preferences, as well.  While 18-moth-olds demonstrate sensitivity to explicit preferences (``Eww!'' to crackers and ``Mmm!'' to broccoli) and comply to give the agent her preferred snack (broccoli, unlike the child's own preference) \cite{repacholi1997}, even 16-month-olds can use statistical information to infer preference. For example, babies who see an experimenter sample five boring toys (white blocks) from a box of mostly boring toys still assume she would prefer the more interesting toys (colorful slinking) if possible, but if they instead see the experimenter sampling the five boring toys from a box of mostly interesting toys, then they expect that she prefers the boring toys to the colorful toys \cite{ma2011}.  Young children are able to consider the likelihood of events and outcomes based on the evidence provided by social partners. 

In addition, two new studies suggest that substantially younger children---some as young as three years old---can make pragmatic inferences in simple referential contexts. In one study, children saw sets of objects (for example, a face, a face with glasses, and a face with a hat and glasses) and heard descriptions like ``my friend has glasses.'' Children in this case were able to make the inference that 


In the absence of difficult linguistic demands, even three-year-olds can make pragmatic implicatures that are grounded in simple referential contexts \cite{stiller2014}, reflecting their ability to reason pragmatically about the generating causes of speakers' actions. 